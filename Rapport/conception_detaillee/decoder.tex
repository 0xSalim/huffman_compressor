
\begin{algorithme}

\procedure
{decoder}
{\paramEntreeSortie{dest : Fichier Biniare }\paramEntree{arbre : Arbre de Huffman, source : Fichier Binaire, longueur : \naturel}}
{bitRestant, i : \naturel, o : octet, cTemp, code : Code Biniare}
{
\affecter{bitRestant}{longueur}
\affecter{cTemp}{CBcreerCodeBinaireVide()}
\affecter{code}{CBcreerCodeBinaireVide()}
\tantque{non (FBfinFichier(source)) ET FBlireOctet(source,o) ET (bitRestants>0))}
{
\affecter{cTemp}{octet\_en\_CB(o)}
\instruction{CBconcatener(code,cTemp)}
\tantque{(CBobtenirLongueur(code)$\geq$NBMAXBITS) ET (bitRestants$>$0)}
{
\instruction{traiterCodeBinaire(code,o,bitRestants,arbre)}
\instruction{FBecrireOctet(dest,o)}
}
}
}


\procedure
{traiterCodeBinaire}
{\paramEntreeSortie{code : Code Binaire, oct : octet, nb : \naturel}\paramEntree{arbre : Arbre de Huffman}}
{i, l : \naturel, aTemp : Arbre de Huffman}
{
\affecter{l}{ CBobtenirLongueur(code)}
\affecter{aTemp}{arbre}
\sialors{l>nb}
{
\pour{i}{l}{nb}{}
{
\instruction{CBsupprimerBit(code,i)}
}}
\tantque{non ADHestUneFeuille(atemp)}
{
\affecter{nb}{nb -1}
\sialorssinon{bitEgaux(CBobtenirBit(code,b0)}
{
\affecter{aTemp}{ADHobtenirFilsGauche(atemp)}
}
{
\affecter{aTemp}{ADHobtenirFilsDroit(atemp)}
}
\instruction{CBsupprimerTete(code)}
}
\instruction{copierOctet(ADHobtenirPere(atemp).valeurOctet,oct)}
}

\fonction
{octetEnCB}
{oct : octet}
{Code Binaire}
{i : \naturel, res : Code Binaire}
{
\affecter{i}{0}
\affecter{res}{CBcreerCodeBinaireVide()}
\pour{i}{0}{8}{}
{
\instruction{CBajouterAlaFin(res,obteniriemeBit(oct,i))}
}
\retourner{res}
}
\end{algorithme}
