Dans le cadre du cours "Algorithmique avanc�e et programmation C", nous devions concevoir un compresseur utilisant l'algorithme de Huffman. Cette m�thode consiste en l'utilisation d'un code � longueur variable pour repr�senter un symbole du fichier source. Le code est d�termin� � partir de la fr�quence d'apparition des symboles de source, un code court �tant associ� aux symboles de source les plus fr�quents. Ainsi, un symbole fr�quent � un code plus court qu'un caract�re n'apparaissant qu'un nombre limit� de fois.
Pour concevoir cet algorithme, nous avons utilis� un arbre binaire dont les feuilles sont des �l�ments du fichier source, ainsi le code correspondant � chaque caract�re est le chemin permettant d?atteindre ces �l�ments.

Notre groupe �tant form� de 5 personnes (Assvin, Leonard, Salim, Thomas et Sylvie), nous avons d� mettre en pratique nos comp�tences et connaissances afin de mener � bien ce projet d'une dur�e de 10 semaines. Dans ce rapport, nous expliciterons donc les �tapes principales du projet, en suivant un cycle en V, en allant de l'analyse jusqu'aux tests.
