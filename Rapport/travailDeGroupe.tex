\section {Difficult�s rencontr�es et am�liorations possible}


\subsection {Probl�mes techniques}

Lors de la r�alisation de notre projet nous avons du d�boguer le programme principal � de nombreuses reprises du � des fuites de m�moires et d'autre erreurs. Nous avons tout de m�me r�ussi � corriger les erreurs gr�ce au d�bogueur Valgrind et � rendre le programme fonctionnel pour des fichiersd'une certaine taille. La compression fonctionne parfaitement, en revanche la d�compression ne fonctionne pas correctement pour des fichiers trop lourd. Un manque de temps ne nous a pas permis de corriger cela malheureusement.
 

\subsection {Organisation du travail}

La difficult� principale rencontr�e dans un projet de groupe est d'arriver � se mettre d'accord sur la conception � choisir. En effet, les choix faits lors de la conception pr�liminaire des TAD devait, � d�faut de correspondre pr�cis�ment � l'id�e que chacun s'en faisait, �tre compris par tous et �tre issus d'une concertation permettant d'�tre le plus proche de la vision de la majorit� du groupe. Il est ainsi n�cessaire de savoir travailler en collaboration et avoir une bonne communication.
C'�tait notamment le cas lorsque nous avons d�cid� d'utiliser le TAD Liste Chain�e en tant que TAD g�n�rique.